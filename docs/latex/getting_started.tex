\hypertarget{getting_started_quickstart}{}\section{Quickstart}\label{getting_started_quickstart}
For the impatient. Must haves\+:~\newline

\begin{DoxyItemize}
\item git
\item C++/\+F90 Compiler
\item C\+Make
\item M\+PI
\item Paralell H\+D\+F5
\end{DoxyItemize}

Optional\+:~\newline

\begin{DoxyItemize}
\item Doxygen (for documentation)
\item Cantera (for reactive flow)
\end{DoxyItemize}

Get, build, and test\+:~\newline
 \begin{quote}
git clone --recursive \href{mailto:git@bitbucket.org}{\tt git@bitbucket.\+org}\+:/xpacc-\/dev/\+Plas\+Com2~\newline
mkdir Plas\+Com2/build~\newline
cd Plas\+Com2/build~\newline
cmake ../~\newline
make~\newline
make test~\newline
\end{quote}


Read on as needed for details.\hypertarget{getting_started_obtain}{}\section{Getting the code}\label{getting_started_obtain}
Plas\+Com2 can be obtained from the \href{https://bitbucket.org:/xpacc-dev/PlasCom2}{\tt X\+P\+A\+CC Bitbucket repository}. Make sure to use the \char`\"{}-\/-\/recursive\char`\"{} option when cloning the repository so that all sub-\/modules will be picked up. For example\+:~\newline
 \begin{quote}
git clone --recursive \href{mailto:git@bitbucket.org}{\tt git@bitbucket.\+org}\+:/xpacc-\/dev/\+Plas\+Com2 \mbox{[}P\+C2\+S\+R\+C\+P\+A\+TH\mbox{]}~\newline
\end{quote}


The above command should create a clone of the Plas\+Com2 repository in at your local path \mbox{[}P\+C2\+S\+R\+C\+P\+A\+TH\mbox{]}. If \mbox{[}P\+C2\+S\+R\+C\+P\+A\+TH\mbox{]} argument is not supplied, then the default path will be ./\+Plas\+Com2. The path to your clone of Plas\+Com2 will hereafter be referred to as P\+C2\+S\+R\+C\+P\+A\+TH.

\begin{DoxyNote}{Note}
Sometimes Plas\+Com2 fails to clone unless the user has S\+SH keys set up with Bitbucket. If there is trouble cloning, please try setting up S\+SH key-\/based access.
\end{DoxyNote}
\hypertarget{getting_started_build}{}\section{Building Plas\+Com2}\label{getting_started_build}
\hypertarget{getting_started_prereq}{}\subsection{Prerequisites}\label{getting_started_prereq}

\begin{DoxyItemize}
\item C++/\+F90 compilers~\newline
 Most modern C++ and F90 compilers should work. Most commonly used and tested are G\+CC, Intel, L\+L\+V\+M/\+Clang/\+Flang, and I\+BM. Plas\+Com2 currently requires only C++98, although should build without issue against C++11.
\item C\+Make 2.\+8 or higher
\item M\+PI~\newline
 Currently only M\+P\+I1 is required, although M\+P\+I2 is likely on the horizon. Plas\+Com2 should build OK against nearly any flavor of M\+PI that implements M\+P\+I1. Most commonly used and tested are M\+P\+I\+CH, Open\+M\+PI, and M\+V\+A\+P\+I\+CH.
\item H\+D\+F5~\newline
 Plas\+Com2 uses parallel H\+D\+F5 for all heavy-\/lifting I/O and requires at least H\+D\+F5-\/1.\+8.\+20. H\+D\+F5 should be built with the M\+PI compiler wrappers with parallel enabled (--enable-\/parallel). If H\+D\+F5 is installed in user space (i.\+e. not in a system-\/wide location), then Plas\+Com2 will need to be made aware of its location at configuration time.
\item Cantera~\newline
 This packages is only required if flow chemistry or combustion is enabled. Refer to Cantera build/install instructions for the process of getting and building it.
\end{DoxyItemize}\hypertarget{getting_started_configbuild}{}\subsection{Configuration and Compiling}\label{getting_started_configbuild}
Plas\+Com2 uses C\+Make (2.\+8+) for configuration, and build management. It is highly recommended to create a build directory that is separate from your Plas\+Com2 source path (P\+C2\+S\+R\+C\+P\+A\+TH). Typically, this is done by creating a build directory below the P\+C2\+S\+R\+C\+P\+A\+TH. For example\+:~\newline
 \begin{quote}
mkdir P\+C2\+S\+R\+C\+P\+A\+T\+H/build \&\& cd P\+C2\+S\+R\+C\+P\+A\+T\+H/build~\newline
\end{quote}


Regardless of where the build will be conducted, the build directory will hereafter be referred to as P\+C2\+B\+L\+D\+P\+A\+TH. There are a few environment variables that can be helpful when configuring Plas\+Com2. If you do not set these environment variables, C\+Make will attempt to find the appropriate setup by searching your environment and common system paths. Plas\+Com2 requires M\+PI, and to ensure the correct building environment is found, it can be useful to set the following\+:~\newline
 \begin{quote}
CC=mpicc~\newline
C\+XX=mpicxx~\newline
FC=mpif90~\newline
\end{quote}


If used, the above environment variables should be set to the desired M\+PI compilers before invoking C\+Make to configure Plas\+Com2. In addition, if third-\/party packages (e.\+g. H\+D\+F5 and/or Cantera) are installed in non-\/standard, or user-\/owned file spaces, then C\+Make must be made aware of the path to those packages. The following environment variable can be used to indicate third-\/party or non-\/standard paths\+:~\newline
 \begin{quote}
C\+M\+A\+K\+E\+\_\+\+P\+R\+E\+F\+I\+X\+\_\+\+P\+A\+TH=/path/to/hdf5\+:/path/to/cantera~\newline
\end{quote}


C\+Make will search the bin, lib, share, and include subdirectories of any paths included in C\+M\+A\+K\+E\+\_\+\+P\+R\+E\+F\+I\+X\+\_\+\+P\+A\+TH for tools, libraries, and include files required by the build.\hypertarget{getting_started_test}{}\section{Testing}\label{getting_started_test}
